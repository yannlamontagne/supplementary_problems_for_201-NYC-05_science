%-----------------------------------------------------
% index key words
%-----------------------------------------------------
\index{subspace}
\index{vector space!subspace}


%-----------------------------------------------------
% name, leave blank
% title, if the exercise has a name i.e. Hilbert's matrix
% difficulty = n, where n is the number of stars
% origin = "\cite{ref}"
%-----------------------------------------------------
\begin{Exercise}[
name={},
title={}, 
difficulty=0,
origin={\cite{JH}}]
Subspaces are subsets and so we naturally consider how `is a subspace of'
interacts with the usual set operations.
\Question If \( A,B \) are subspaces of a vector space, are
        their intersection
        \( A\intersection B \) be a subspace?
\Question Is the union \( A\union B \) a subspace?
\Question If \( A \) is a subspace, is its complement be a subspace?

\end{Exercise}

\begin{Answer}
        \Question Is is a subspace.

          Assume that \( A,B \) are subspaces of \( V \).
          Note that 
          their intersection is not empty as both contain the zero vector.
          If \( \vec{w},\vec{s}\in A\intersection B \) and \( r,s \) are
          scalars then \( r\vec{v}+s\vec{w}\in A \) because
          each vector is in \( A \) and so a linear combination is in \( A \),
          and \(r\vec{v}+s\vec{w}\in B \) for the same reason.
          Thus the intersection is closed.
        \Question In general it is not a subspace.  (It is a subspace, only if \( A\subseteq B \) or
          \( B\subseteq A \)).

          Take \( V \) to be \( \Re^3 \),
          take \( A \) to be the $x$-axis, and \( B \) to be the
          \( y \)-axis.
          Note that
          \begin{equation*}
            \colvec[r]{1 \\ 0}\in A \text{ and }\colvec[r]{0 \\ 1}\in B
            \quad\text{but}\quad
            \colvec[r]{1 \\ 0}+\colvec[r]{0 \\ 1}\not\in A\union B
          \end{equation*}
          as the sum is in neither \( A \) nor \( B \).

          If \( A\subseteq B \) or
          \( B\subseteq A \) then clearly \( A\union B \) is a subspace.

          To show that \( A\union B \) is a subspace only if one
          subspace contains the other, we assume that \( A\not\subseteq B \)
          and \( B\not\subseteq A \) and prove that 
          the union is not a subspace.
          The assumption that \( A \) is not a subset of \( B \) means that 
          there is an \( \vec{a}\in A \) with \( \vec{a}\not\in B \).
          The other assumption gives a \( \vec{b}\in B \) with
          \( \vec{b}\not\in A \).
          Consider \( \vec{a}+\vec{b} \).
          Note that sum is not an element of \( A \) or else
          \( (\vec{a}+\vec{b})-\vec{a} \) would be in \( A \), which it is not.
          Similarly the sum is not an element of \( B \).
          Hence the sum is not an element of \( A\union B \), and so the union
          is not a subspace.
 \Question It is not a subspace.
          As \( A \) is a subspace, it contains the zero vector, and therefore
          the set that is $A$'s complement does not.
          Without the zero vector, the complement cannot be a vector space.


\end{Answer}
