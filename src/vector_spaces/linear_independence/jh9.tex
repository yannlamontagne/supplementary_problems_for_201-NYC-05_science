%-----------------------------------------------------
% index key words
%-----------------------------------------------------
\index{linear independence}
\index{linear dependence}

%-----------------------------------------------------
% name, leave blank
% title, if the exercise has a name i.e. Hilbert's matrix
% difficulty = n, where n is the number of stars
% origin = "\cite{ref}"
%-----------------------------------------------------
\begin{Exercise}[
name={},
title={}, 
difficulty=0,
origin={\cite{JH}}]
\Question Show that this 
        \begin{equation*}
          S=\set{\colvec{1 \\ 1 \\ 0},\colvec{-1 \\ 2 \\ 0}}
        \end{equation*}
        is a linearly independent subset of \( \Re^3 \).
\Question Show that
        \begin{equation*}
          \colvec{3 \\ 2 \\ 0}
        \end{equation*}
        is in the span of $S$ by finding \( c_1 \) and \( c_2 \) 
        giving a linear relationship.
        \begin{equation*}
          c_1\colvec{1 \\ 1 \\ 0}
          +c_2\colvec{-1 \\ 2 \\ 0}
          =\colvec{3 \\ 2 \\ 0}
        \end{equation*}
        Show that the pair \( c_1,c_2 \) is unique.
\Question Assume that \( S \) is a subset of a vector space and that
        \( \vec{v} \) is in \( \spanof{S} \), so that \( \vec{v} \) is a
        linear combination of vectors from \( S \).
        Prove that if \( S \) is linearly independent then a linear combination
        of vectors from \( S \) adding to \( \vec{v} \)
        is unique (that is, unique up to reordering
        and adding or taking away terms of the form \( 0\cdot\vec{s} \)).
        Thus \( S \) as a spanning set is minimal in this strong sense:
        each vector in \( \spanof{S} \) is a combination of elements
        of $S$ a minimum number of
        times (only once).
\Question
        Prove that it can happen when \( S \) is not linearly 
        independent that distinct linear combinations sum to the same vector.


\end{Exercise}

\begin{Answer}
\Question The linear system arising from
          \begin{equation*}
            c_1\colvec{1 \\ 1 \\ 0}
            +c_2\colvec{-1 \\ 2 \\ 0}
            =\colvec{0 \\ 0 \\ 0}
          \end{equation*}
          has the unique solution \( c_1=0 \) and \( c_2=0 \).
\Question The linear system arising from
          \begin{equation*}
            c_1\colvec{1 \\ 1 \\ 0}
            +c_2\colvec{-1 \\ 2 \\ 0}
            =\colvec{3 \\ 2 \\ 0}
          \end{equation*}
          has the unique solution \( c_1=8/3 \) and \( c_2=-1/3 \).
\Question Suppose that \( S \) is linearly independent.
          Suppose that we have both $\vec{v}=c_1\vec{s}_1+\dots+c_n\vec{s}_n$
          and $\vec{v}=d_1\vec{t}_1+\dots+d_m\vec{t}_m$
          (where the vectors are members of $S$).
          Now, 
          \begin{equation*}
            c_1\vec{s}_1+\dots+c_n\vec{s}_n
            =\vec{v}
            =d_1\vec{t}_1+\dots+d_m\vec{t}_m
          \end{equation*}
          can be rewritten in this way.
          \begin{equation*}
            c_1\vec{s}_1+\dots+c_n\vec{s}_n
            -d_1\vec{t}_1-\dots-d_m\vec{t}_m
            =\zero
          \end{equation*}
          Possibly some of the $\vec{s}\,$'s equal some of the $\vec{t}\,$'s;
          we can combine the associated coefficients 
          (i.e., if $\vec{s}_i=\vec{t}_j$ then
          $\cdots+c_i\vec{s}_i+\dots-d_j\vec{t}_j-\cdots$ can be rewritten
          as $\cdots+(c_i-d_j)\vec{s}_i+\cdots$).
          That equation is a linear relationship among  
          distinct (after the combining is done) members of the set $S$.
          We've assumed that $S$ is linearly independent, so all of the 
          coefficients are zero.
          If $i$ is such that $\vec{s}_i$ does not equal any $\vec{t}_j$
          then $c_i$ is zero.
          If $j$ is such that $\vec{t}_j$ does not equal any $\vec{s}_i$
          then $d_j$ is zero.
          In the final case, we have that $c_i-d_j=0$ and so $c_i=d_j$.  

          Therefore, the original two sums are the same, except perhaps for
          some $0\cdot\vec{s}_i$ or $0\cdot\vec{t}_j$ terms that we can
          neglect.
\Question
          This set is not linearly independent:
          \begin{equation*}
            S=\set{\colvec{1 \\ 0},\colvec{2 \\ 0}}\subset\Re^2
          \end{equation*}
          and these two linear combinations give the same result
          \begin{equation*}
            \colvec{0 \\ 0}=2\cdot\colvec{1 \\ 0}-1\cdot\colvec{2 \\ 0}
                            =4\cdot\colvec{1 \\ 0}-2\cdot\colvec{2 \\ 0}
          \end{equation*}
          Thus, a linearly dependent set might have indistinct sums.

          In fact, this stronger statement holds:~if a set is linearly 
          dependent then it must have the property that there are two 
          distinct linear combinations that sum to the same vector.
          Briefly, where \( c_1\vec{s}_1+\dots+c_n\vec{s}_n=\zero \) then
          multiplying both sides of the relationship by two gives another 
          relationship.
          If the first relationship is nontrivial then the second is also.

\end{Answer}
