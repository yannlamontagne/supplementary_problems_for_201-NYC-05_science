%-----------------------------------------------------
% index key words
%-----------------------------------------------------
\index{linear independence}
\index{linear dependence}

%-----------------------------------------------------
% name, leave blank
% title, if the exercise has a name i.e. Hilbert's matrix
% difficulty = n, where n is the number of stars
% origin = "\cite{ref}"
%-----------------------------------------------------
\begin{Exercise}[
name={},
title={}, 
difficulty=0,
origin={\cite{JH}}]
\Question Show that any set of four vectors in \( \Re^2 \) is 
         linearly dependent.
\Question Is this true for any set of five?
         Any set of three?
\Question What is the most number of elements that a 
         linearly independent subset of $\Re^2$ can have?

\end{Exercise}

\begin{Answer}
\Question For any $a_{1,1}$, \ldots, $a_{2,4}$,
          \begin{equation*}
            c_1\colvec{a_{1,1} \\ a_{2,1}}
            +c_2\colvec{a_{1,2} \\ a_{2,2}}
            +c_3\colvec{a_{1,3} \\ a_{2,3}}
            +c_4\colvec{a_{1,4} \\ a_{2,4}}
            =\colvec{0 \\ 0}
          \end{equation*}
          yields a linear system
          \begin{equation*}
             \begin{linsys}{4}
              a_{1,1}c_1 &+ &a_{1,2}c_2 &+ &a_{1,3}c_3 &+ &a_{1,4}c_4 &= &0  \\
              a_{2,1}c_1 &+ &a_{2,2}c_2 &+ &a_{2,3}c_3 &+ &a_{2,4}c_4 &= &0  
             \end{linsys}
          \end{equation*}
        that has infinitely many solutions (Gauss's Method leaves at least
        two variables free).
        Hence there are nontrivial linear relationships among the given
        members of $\Re^2$.
\Question Any set five vectors is a superset of a set of four vectors,
        and so is linearly dependent.

        With three vectors from $\Re^2$, the argument from the prior item 
        still applies, with the slight change that Gauss's Method now only 
        leaves at least one variable free (but that still gives infinitely many
        solutions).
\Question The prior part shows that no three-element subset of $\Re^2$
        is independent.
        We know that there are two-element subsets of $\Re^2$ that are 
        independent. The following one is
        \begin{equation*}
          \set{\colvec{1  \\ 0},\colvec{0  \\ 1}}
        \end{equation*} 
        and so the answer is two.

\end{Answer}
