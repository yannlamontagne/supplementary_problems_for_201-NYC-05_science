%-----------------------------------------------------
% index key words
%-----------------------------------------------------
\index{matrix space}
\index{dimension}

%-----------------------------------------------------
% name, leave blank
% title, if the exercise has a name i.e. Hilbert's matrix
% difficulty = n, where n is the number of stars
% origin = "\cite{ref}"
%-----------------------------------------------------
\begin{Exercise}[
name={},
title={}, 
difficulty=0,
origin={\cite{JH}}]
Find the dimension of the vector space of matrices
\begin{equation*}
\begin{mat}
a  &b  \\
c  &d
\end{mat}
\end{equation*}
subject to each condition.
\Question $a, b, c, d\in\Re$ 
\Question $a-b+2c=0$ and~$d\in\Re$
\Question $a+b+c=0$, $a+b-c=0$, and~$d\in\Re$
\end{Exercise}

\begin{Answer}
\Question As in the prior exercise, the space $\matspace_{\nbyn{2}}$ 
        of matrices without restriction has this basis
        \begin{equation*}
         \sequence{
           \begin{mat}[r]
             1  &0  \\
             0  &0
           \end{mat},
           \begin{mat}[r]
             0  &1  \\
             0  &0
           \end{mat},
           \begin{mat}[r]
             0  &0  \\
             1  &0
           \end{mat},
           \begin{mat}[r]
             0  &0  \\
             0  &1
           \end{mat}  }
        \end{equation*}
        and so the dimension is four.
\Question For this space
        \begin{multline*}
         \set{\begin{mat}
               a  &b  \\
               c  &d
             \end{mat} \suchthat \text{$a=b-2c$ and $d\in\Re$}}     \\
         =\set{b\cdot\begin{mat}[r]
             1  &1  \\
             0  &0
           \end{mat}
           +c\cdot\begin{mat}[r]
             -2  &0  \\
              1  &0
           \end{mat}
           +d\cdot\begin{mat}[r]
             0  &0  \\
             0  &1
           \end{mat} \suchthat b,c,d\in\Re}
        \end{multline*}
        this is a natural basis.
        \begin{equation*}
          \sequence{
            \begin{mat}[r]
              1  &1  \\
              0  &0
            \end{mat},
            \begin{mat}[r]
              -2  &0  \\
               1  &0
            \end{mat},
            \begin{mat}[r]
              0  &0  \\
              0  &1
            \end{mat}  }
        \end{equation*}
        The dimension is three. 
\Question Gauss's Method applied to the two-equation linear system gives
        that $c=0$ and that $a=-b$.
        Thus, we have this description
        \begin{equation*}
         \set{\begin{mat}
               -b  &b  \\
                0  &d
             \end{mat} \suchthat b,d\in\Re}
         =\set{b\cdot\begin{mat}[r]
             -1  &1  \\
             0  &0
           \end{mat}
           +d\cdot\begin{mat}[r]
             0  &0  \\
             0  &1
           \end{mat} \suchthat b,d\in\Re}
        \end{equation*}
        and so this is a natural basis.
        \begin{equation*}
          \sequence{
            \begin{mat}[r]
              -1  &1  \\
               0  &0
            \end{mat},
            \begin{mat}[r]
              0  &0  \\
              0  &1
            \end{mat}  }
        \end{equation*}
        The dimension is two. 
\end{Answer}
