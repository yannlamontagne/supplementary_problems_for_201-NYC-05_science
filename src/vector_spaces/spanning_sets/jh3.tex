%-----------------------------------------------------
% index key words
%-----------------------------------------------------
\index{subspace}
\index{spaning set}
%-----------------------------------------------------
% name, leave blank
% title, if the exercise has a name i.e. Hilbert's matrix
% difficulty = n, where n is the number of stars
% origin = "\cite{ref}"
%-----------------------------------------------------
\begin{Exercise}[
name={},
title={}, 
difficulty=0,
origin={\cite{JH}}]
Which of these sets spans \( \Re^3 \)?
\Question \( \set{ \colvec[r]{1 \\ 0 \\ 0},
               \colvec[r]{0 \\ 2 \\ 0},
               \colvec[r]{0 \\ 0 \\ 3}  } \)
\Question \( \set{ \colvec[r]{2 \\ 0 \\ 1},
               \colvec[r]{1 \\ 1 \\ 0},
               \colvec[r]{0 \\ 0 \\ 1}  } \)
\Question \( \set{ \colvec[r]{1 \\ 1 \\ 0},
               \colvec[r]{3 \\ 0 \\ 0}  } \)
\Question \( \set{ \colvec[r]{1 \\ 0 \\ 1},
               \colvec[r]{3 \\ 1 \\ 0},
               \colvec[r]{-1 \\ 0 \\ 0},
               \colvec[r]{2 \\ 1 \\ 5}  } \)
\Question \( \set{ \colvec[r]{2 \\ 1 \\ 1},
               \colvec[r]{3 \\ 0 \\ 1},
               \colvec[r]{5 \\ 1 \\ 2},
               \colvec[r]{6 \\ 0 \\ 2}  } \)
\end{Exercise}

\begin{Answer}
         \Question Yes, for any \( x,y,z\in\Re \) this equation
           \begin{equation*}
              r_1\colvec[r]{1 \\ 0 \\ 0}
              +r_2\colvec[r]{0 \\ 2 \\ 0}
              +r_3\colvec[r]{0 \\ 0 \\ 3}
              =\colvec{x \\ y \\ z}
           \end{equation*}
           has the solution \( r_1=x \), \( r_2=y/2 \), and
           \( r_3=z/3 \).
         \Question Yes, the equation
           \begin{equation*}
             r_1\colvec[r]{2 \\ 0 \\ 1}
             +r_2\colvec[r]{1 \\ 1 \\ 0}
             +r_3\colvec[r]{0 \\ 0 \\ 1}
             =\colvec{x \\ y \\ z}
           \end{equation*}
           gives rise to this 
           \begin{equation*}
             \begin{linsys}{3}
               2r_1 &+  &r_2  &  &    &=  &x \\
                    &   &r_2  &  &    &=  &y \\
                r_1 &   &     &+ &r_3 &=  &z \\
             \end{linsys}
	\end{equation*}
        Gaussian elimination gives
	\begin{equation*} 
	\begin{linsys}{3}
               2r_1 &+  &r_2  &  &    &=  &x\hfill\hbox{} \\
                    &   &r_2  &  &    &=  &y\hfill\hbox{} \\
                    &   &     &  &r_3 &=  &-(1/2)x+(1/2)y+z \\
             \end{linsys}
           \end{equation*}
           so that, given any $x$, $y$, and $z$, we can compute that
           \( r_3=(-1/2)x+(1/2)y+z \), \( r_2=y \), and
           \( r_1=(1/2)x-(1/2)y \).
        \Question No.
           In particular, we cannot get the vector
           \begin{equation*}
             \colvec[r]{0 \\ 0 \\ 1}
           \end{equation*}
           as a linear combination since the two given
           vectors both have a third component of zero.
       \Question Yes.
         The equation
         \begin{equation*}
           r_1\colvec[r]{1 \\ 0 \\ 1}
           +r_2\colvec[r]{3 \\ 1 \\ 0}
           +r_3\colvec[r]{-1\\ 0 \\ 0}
           +r_4\colvec[r]{2 \\ 1 \\ 5}
           =\colvec{x \\ y \\ z}
         \end{equation*}
         leads to this reduction.
         \begin{equation*}
           \begin{amat}{4}
             1  &3  &-1  &2  &x \\
             0  &1  &0   &1  &y  \\
             0  &0  &1   &6  &-x+3y+z
           \end{amat}
         \end{equation*}
         We have infinitely many solutions.
         We can, for example, set $r_4$ to be zero and solve for
         $r_3$, $r_2$, and $r_1$ in terms of $x$, $y$, and $z$ by the usual
         methods of back-substitution.
       \Question No.
         The equation
         \begin{equation*}
           r_1\colvec[r]{2 \\ 1 \\ 1}
           +r_2\colvec[r]{3 \\ 0 \\ 1}
           +r_3\colvec[r]{5 \\ 1 \\ 2}
           +r_4\colvec[r]{6 \\ 0 \\ 2}
           =\colvec{x \\ y \\ z}
         \end{equation*}
         leads to this reduction.
         \begin{multline*}
           \begin{amat}{4}
             2  &3     &5     &6  &x \\
             0  &-3/2  &-3/2  &-3 &-(1/2)x+y  \\
             0  &0     &0     &0  &-(1/3)x-(1/3)y+z
           \end{amat}
         \end{multline*}
         This shows that not every vector can be so expressed.
         Only the vectors satisfying the restriction that
         $-(1/3)x-(1/3)y+z=0$ are in the span.
         (To see that any such vector is indeed expressible, 
         take $r_3$ and $r_4$
         to be zero and solve for $r_1$ and $r_2$ in terms of $x$, $y$, and
         $z$ by back-substitution.)


\end{Answer}
