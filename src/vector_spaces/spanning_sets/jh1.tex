%-----------------------------------------------------
% index key words
%-----------------------------------------------------
\index{subspace}
\index{spaning set}
\index{vector space}
\index{polynomial space}
\index{matrix space}

%-----------------------------------------------------
% name, leave blank
% title, if the exercise has a name i.e. Hilbert's matrix
% difficulty = n, where n is the number of stars
% origin = "\cite{ref}"
%-----------------------------------------------------
\begin{Exercise}[
name={},
title={}, 
difficulty=0,
origin={\cite{JH}}]
    Determine whether the vector lies in the span of the set.
\Question \( \colvec[r]{2 \\ 0 \\ 1} \),
        \( \set{\colvec[r]{1 \\ 0 \\ 0},
                \colvec[r]{0 \\ 0 \\ 1}  } \)
\Question \( x-x^3 \),
        \( \set{x^2,2x+x^2,x+x^3} \)
\Question \( \begin{mat}[r]
                 0  &1  \\
                 4  &2
               \end{mat}  \),
        \( \set{\begin{mat}[r]
                  1  &0  \\
                  1  &1
                \end{mat},
                \begin{mat}[r]
                  2  &0  \\
                  2  &3
                \end{mat}  } \)

\end{Exercise}

\begin{Answer}
\Question Yes, solving the linear system arising from
           \begin{equation*}
             r_1\colvec[r]{1 \\ 0 \\ 0}+r_2\colvec[r]{0 \\ 0 \\ 1}
               =\colvec[r]{2 \\ 0 \\ 1}
           \end{equation*}
           gives \( r_1=2 \) and \( r_2=1 \).
\Question Yes; the linear system arising from
           \( r_1(x^2)+r_2(2x+x^2)+r_3(x+x^3)=x-x^3 \)
           \begin{equation*}
             \begin{linsys}{3}
                   &  &2r_2 &+ &r_3 &= &1  \\
               r_1 &+ &r_2  &  &    &= &0  \\
                   &  &     &  &r_3 &= &-1   
             \end{linsys}
           \end{equation*}
           gives that \( -1(x^2)+1(2x+x^2)-1(x+x^3)=x-x^3 \).
\Question No; any combination of the two given matrices has a zero
           in the upper right.

\end{Answer}
