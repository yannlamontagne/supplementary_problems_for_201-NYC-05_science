%-----------------------------------------------------
% index key words
%-----------------------------------------------------
\index{system of linear equations}
\index{Kirchhoff's law}

%-----------------------------------------------------
% name, leave blank
% title, if the exercise has a name i.e. Hilbert's matrix
% difficulty = n, where n is the number of stars
% origin = "\cite{ref}"
%-----------------------------------------------------
\begin{Exercise}[
name={},
title={}, 
difficulty=0,
origin={\cite{KK}}]
Consider the following diagram of three circuits.

\begin{picture}(1,155)
\put(100,10){
%\put(0,0){\Line(0,0)(0,30)}\put(0,30){\thicklines \qbezier(-3,0)(0,0)(3,0)}
%\put(0,33){\qbezier(-8,0)(0,0)(8,0)}
%\Line(0,33)(0,60)(20,60)
\Line(0,60)(20,60)
\put(20,60){\thicklines \setlength{\unitlength}{.5pt} \Line(0,0)(3,7)(9,-7)(15,7)(21,-7)(27,7)(33,-7)(39,7)(45,-7)(48,0)}
\Line(44,60)(60,60)(60,40)
\put(60,16){\thicklines \setlength{\unitlength}{.5pt} \Line(0,0)(7,3)(-7,9)(7,15)(-7,21)(7,27)(-7,33)(7,39)(-7,45)(0,48)}
%\Line(0,0)(20,0)\put(20,0){\thicklines \setlength{\unitlength}{.5pt} \Line(0,0)(3,7)(9,-7)(15,7)(21,-7)(27,7)(33,-7)(39,7)(45,-7)(48,0)}
%\Line(44,0)(60,0)(60,16)
\Line(60,0)(60,16)
\put(0,60){\Line(0,0)(0,30)\put(0,33){\thicklines \qbezier(-3,0)(0,0)(3,0)}\put(0,30){\qbezier(-8,0)(0,0)(8,0)}\Line(0,33)(0,60)(20,60)}
\put(20,120){\thicklines \setlength{\unitlength}{.5pt} \Line(0,0)(3,7)(9,-7)(15,7)(21,-7)(27,7)(33,-7)(39,7)(45,-7)(48,0)}
\Line(44,120)(60,120)(60,100)
\put(60,76){\thicklines \setlength{\unitlength}{.5pt} \Line(0,0)(7,3)(-7,9)(7,15)(-7,21)(7,27)(-7,33)(7,39)(-7,45)(0,48)}
\Line(60,76)(60,60)
\Line(60,120)(80,120)\put(80,120){\thicklines \qbezier(0,-3)(0,0)(0,3)}
\put(83,120){\qbezier(0,-8)(0,0)(0,8)}
\Line(83,120)(96,120)
\put(96,120){\thicklines \setlength{\unitlength}{.5pt} \Line(0,0)(3,7)(9,-7)(15,7)(21,-7)(27,7)(33,-7)(39,7)(45,-7)(48,0)}
\Line(60,60)(80,60)
\put(80,60){\thicklines \setlength{\unitlength}{.5pt} \Line(0,0)(3,7)(9,-7)(15,7)(21,-7)(27,7)(33,-7)(39,7)(45,-7)(48,0)}
\Line(104,60)(120,60)(120,80)
\put(120,80){\thicklines \setlength{\unitlength}{.5pt} \Line(0,0)(7,3)(-7,9)(7,15)(-7,21)(7,27)(-7,33)(7,39)(-7,45)(0,48)}
\Line(120,104)(120,120)
\Line(60,0)(80,0)
\put(80,0){\thicklines \setlength{\unitlength}{.5pt} \Line(0,0)(3,7)(9,-7)(15,7)(21,-7)(27,7)(33,-7)(39,7)(45,-7)(48,0)}
\Line(104,0)(120,0)(120,20)
\put(120,20){\thicklines \setlength{\unitlength}{.5pt} \Line(0,0)(7,3)(-7,9)(7,15)(-7,21)(7,27)(-7,33)(7,39)(-7,45)(0,48)}
\Line(120,44)(120,60)
%\put(-40,30){$10\  volts$}
\put(-40,90){$10\  volts$}
\put(60,130){$12\  volts$}
\put(25,130){$3\  \Omega$}
\put(25,70){$2\  \Omega$}
%\put(25,10){$4\  \Omega$}
\put(100,130){$7\ \Omega$}
\put(80,70){$1\  \Omega$}
\put(80,10){$4\  \Omega$}
\put(40,90){$5\  \Omega$}
\put(130,90){$3\  \Omega$}
\put(130,30){$4\  \Omega$}
\put(40,30){$2\  \Omega$}
%\put(20,40){$I_1$}
\put(20,100){$I_1$}
\put(100,100){$I_2$}
\put(100,40){$I_3$}
    }
\end{picture}

The current in
amps in the four circuits is denoted by $I_{1},\;I_{2},\;I_{3}$ and it is
understood that the motion is in the counter clockwise direction.  Solve for $I_{1},\;I_{2},\;I_{3}$.
\end{Exercise}

\begin{Answer}
The equations obtained are
\begin{eqnarray*}
2I_{1}+5I_{1}+3I_{1}-5I_{2} &=&-10 \\
I_{2}+3I_{2}+7I_{2}+5I_{2}-5I_{1} &=&-12 \\
2I_{3}+4I_{3}+4I_{3}+I_{3}-I_{2} &=&0
\end{eqnarray*}
Simplifying this yields
\begin{eqnarray*}
\begin{linsys}{2}
10I_{1}-5I_{2} &=&-10 \\
16I_{2}-5I_{1} &=&-12 \\
11I_{3}-I_{2} &=&0
\end{linsys}
\end{eqnarray*}
Solving the system gives
\[
I_{1}=-
\frac{44}{27},I_{2}=-\frac{34}{27},I_{3}=-\frac{34}{297}
\]
Thus all currents flow in the clockwise direction.
\end{Answer}
