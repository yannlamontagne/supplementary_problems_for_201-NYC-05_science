%-----------------------------------------------------
% index key words
%-----------------------------------------------------
\index{system of linear equations}
\index{Kirchhoff's law}

%-----------------------------------------------------
% name, leave blank
% title, if the exercise has a name i.e. Hilbert's matrix
% difficulty = n, where n is the number of stars
% origin = "\cite{ref}"
%-----------------------------------------------------
\begin{Exercise}[
name={},
title={}, 
difficulty=0,
origin={\cite{KK}}]
Consider the following diagram of four circuits.

\begin{picture}(1,155)
%\setlength{\unitlength}{0.2pt}
\put(100,10){\put(0,0){\Line(0,0)(0,30)}\put(0,30){\thicklines \qbezier(-3,0)(0,0)(3,0)}\put(0,33){\qbezier(-8,0)(0,0)(8,0)}\Line(0,33)(0,60)(20,60)
\put(20,60){\thicklines \setlength{\unitlength}{.5pt} \Line(0,0)(3,7)(9,-7)(15,7)(21,-7)(27,7)(33,-7)(39,7)(45,-7)(48,0)}
\Line(44,60)(60,60)(60,40)
\put(60,16){\thicklines \setlength{\unitlength}{.5pt} \Line(0,0)(7,3)(-7,9)(7,15)(-7,21)(7,27)(-7,33)(7,39)(-7,45)(0,48)}
\Line(0,0)(20,0)\put(20,0){\thicklines \setlength{\unitlength}{.5pt} \Line(0,0)(3,7)(9,-7)(15,7)(21,-7)(27,7)(33,-7)(39,7)(45,-7)(48,0)}
\Line(44,0)(60,0)(60,16)
\put(0,60){\Line(0,0)(0,30)\put(0,33){\thicklines \qbezier(-3,0)(0,0)(3,0)}\put(0,30){\qbezier(-8,0)(0,0)(8,0)}\Line(0,33)(0,60)(20,60)}
\put(20,120){\thicklines \setlength{\unitlength}{.5pt} \Line(0,0)(3,7)(9,-7)(15,7)(21,-7)(27,7)(33,-7)(39,7)(45,-7)(48,0)}
\Line(44,120)(60,120)(60,100)
\put(60,76){\thicklines \setlength{\unitlength}{.5pt} \Line(0,0)(7,3)(-7,9)(7,15)(-7,21)(7,27)(-7,33)(7,39)(-7,45)(0,48)}
\Line(60,76)(60,60)
\Line(60,120)(80,120)\put(80,120){\thicklines \qbezier(0,-3)(0,0)(0,3)}
\put(83,120){\qbezier(0,-8)(0,0)(0,8)}
\Line(83,120)(96,120)
\put(96,120){\thicklines \setlength{\unitlength}{.5pt} \Line(0,0)(3,7)(9,-7)(15,7)(21,-7)(27,7)(33,-7)(39,7)(45,-7)(48,0)}
\Line(60,60)(80,60)
\put(80,60){\thicklines \setlength{\unitlength}{.5pt} \Line(0,0)(3,7)(9,-7)(15,7)(21,-7)(27,7)(33,-7)(39,7)(45,-7)(48,0)}
\Line(104,60)(120,60)(120,80)
\put(120,80){\thicklines \setlength{\unitlength}{.5pt} \Line(0,0)(7,3)(-7,9)(7,15)(-7,21)(7,27)(-7,33)(7,39)(-7,45)(0,48)}
\Line(120,104)(120,120)
\Line(60,0)(80,0)
\put(80,0){\thicklines \setlength{\unitlength}{.5pt} \Line(0,0)(3,7)(9,-7)(15,7)(21,-7)(27,7)(33,-7)(39,7)(45,-7)(48,0)}
\Line(104,0)(120,0)(120,20)
\put(120,20){\thicklines \setlength{\unitlength}{.5pt} \Line(0,0)(7,3)(-7,9)(7,15)(-7,21)(7,27)(-7,33)(7,39)(-7,45)(0,48)}
\Line(120,44)(120,60)
\put(-40,30){$10\  \text{volts}$}
\put(-40,90){$5\  \text{volts}$}
\put(60,130){$20\  \text{volts}$}
\put(25,130){$3\  \Omega$}
\put(25,70){$2\  \Omega$}
\put(25,10){$4\  \Omega$}
\put(100,130){$1\ \Omega$}
\put(80,70){$6\  \Omega$}
\put(80,10){$2\  \Omega$}
\put(40,90){$5\  \Omega$}
\put(130,90){$1\  \Omega$}
\put(130,30){$3\  \Omega$}
\put(40,30){$1\  \Omega$}
\put(20,40){$I_1$}
\put(20,100){$I_2$}
\put(100,100){$I_3$}
\put(100,40){$I_4$}
    }
\end{picture}

The jagged lines denote resistors and the numbers next to them give their
resistance in ohms, written as $\Omega $. The breaks in the lines having one
short line and one long line denote a voltage source which causes the
current to flow in the direction which goes from the longer of the two lines
toward the shorter along the unbroken part of the circuit. The current in
amps in the four circuits is denoted by $I_{1},I_{2},I_{3},I_{4}$ and it is
understood that the motion is in the counter clockwise direction. If $I_{k}$
ends up being negative, then it just means the current flows in the
clockwise direction. Then Kirchhoff's law states:


\textit{The sum of the resistance times the amps in the counter clockwise direction
around a loop equals the sum of the voltage sources in the same direction
around the loop.}

In the above diagram, the top left circuit gives the equation
\begin{equation*}
2I_{2}-2I_{1}+5I_{2}-5I_{3}+3I_{2}=5
\end{equation*}
For the circuit on the lower left,
\begin{equation*}
4I_{1}+I_{1}-I_{4}+2I_{1}-2I_{2}=-10
\end{equation*}
Write equations for each of the other two circuits and then give a solution
to the resulting system of equations. 
\end{Exercise}

\begin{Answer}
The two other equations are
\[
\begin{linsys}{6}
6I_{3}&-&6I_{4}&+&I_{3}&+&I_{3}&+&5I_{3}&-&5I_{2} &=&-20 \\
2I_{4}&+&3I_{4}&+&6I_{4}&-&6I_{3}&+&I_{4}&-&I_{1} &=&0
\end{linsys}
\]
Then the system is 
\[
\begin{linsys}{6}
2I_{2}&-&2I_{1}&+&5I_{2}&-&5I_{3}&+&3I_{2}&=&5 \\
4I_{1}&+&I_{1}&-&I_{4}&+&2I_{1}&-&2I_{2}&=&-10 \\
6I_{3}&-&6I_{4}&+&I_{3}&+&I_{3}&+&5I_{3}&-&5I_{2}&=&-20 \\
2I_{4}&+&3I_{4}&+&6I_{4}&-&6I_{3}&+&I_{4}&-&I_{1}&=&0
\end{linsys}
\]
The solution is:
\[
 I_{1}=-2.\,\allowbreak 010\,7,I_{2}=-1.\,\allowbreak
269\,9,I_{3}=-2.\,\allowbreak 735\,5,I_{4}=-1.\,\allowbreak 535\,3
\]
\end{Answer}
