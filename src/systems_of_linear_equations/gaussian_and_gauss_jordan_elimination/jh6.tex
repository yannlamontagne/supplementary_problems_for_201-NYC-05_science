%-----------------------------------------------------
% index key words
%-----------------------------------------------------
\index{}


%-----------------------------------------------------
% name, leave blank
% title, if the exercise has a name i.e. Hilbert's matrix
% difficulty = n, where n is the number of stars
% origin = "by name \cite{ref}"
%-----------------------------------------------------
\begin{Exercise}[
name={},
title={}, 
difficulty=0,
origin={\cite{JH}}]
Show that if \( ad-bc\neq 0 \) then
\begin{equation*}
\begin{linsys}{2}
ax  &+  &by  &=  &j  \\
cx  &+  &dy  &=  &k  
\end{linsys}
\end{equation*}
has a unique solution.
\end{Exercise}

\begin{Answer}
     We take three cases: that $a\neq 0$, that $a=0$ and 
      $c\neq 0$, and that both $a=0$ and $c=0$.

      For the first, we assume that \( a\neq 0 \).
      Then Gaussian elimination
      \begin{equation*}
        \begin{linsys}{2}
          ax  &+  &by                  &=  &j \hfill\hbox{} \\
              &   &(-(cb/a)+d)y  &=  &-(cj/a)+k \hfill  
         \end{linsys}
      \end{equation*}
      shows that this system has a unique solution if and only if
      \( -(cb/a)+d\neq 0   \); remember that \( a\neq 0 \) so 
      that back substitution yields a unique \( x \)
      (observe, by the way, that \( j \) and \( k \) play no role in the
      conclusion that there is a unique solution, although if there is a 
      unique solution then they contribute to its value).
      But \( -(cb/a)+d = (ad-bc)/a \) and a fraction is not equal to \( 0 \) 
      if and only if its numerator is not equal to \( 0 \).
      Thus, in this first case, there is a unique solution if and only if
      $ad-bc\neq 0$.

      In the second case, if \( a=0 \) but \( c\neq 0 \), then we swap
      \begin{equation*}
        \begin{linsys}{2}
          cx  &+  &dy  &=  &k  \\
              &   &by  &=  &j  
        \end{linsys}
      \end{equation*}
      to conclude that the system has a unique solution if and only if 
      \( b\neq 0 \)
      (we use the case assumption that \( c\neq 0 \) to get a unique
      \( x \) in back substitution).
      But where \( a=0 \) and \( c\neq 0 \)
      the condition ``\( b\neq 0 \)''
      is equivalent to the condition ``\( ad-bc\neq 0 \)''.
      That finishes the second case.

      Finally, for the third case,
      if both \( a \) and \( c \) are \( 0 \) then the system
      \begin{equation*}
        \begin{linsys}{2}
          0x  &+  &by  &=  &j  \\
          0x  &+  &dy  &=  &k  
        \end{linsys}
      \end{equation*}
      might have no solutions (if the second equation is not a multiple of the
      first) or it might have infinitely many solutions (if the second
      equation is a multiple of the first then for each \( y \) satisfying
      both equations, any pair \( (x,y) \) will do), but it never has a unique
      solution.
      Note that \( a=0 \) and \( c=0 \) gives that \( ad-bc=0 \). 
\end{Answer}
