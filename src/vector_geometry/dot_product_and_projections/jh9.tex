%-----------------------------------------------------
% index key words
%-----------------------------------------------------
\index{orthogonal vector}

%-----------------------------------------------------
% name, leave blank
% title, if the exercise has a name i.e. Hilbert's matrix
% difficulty = n, where n is the number of stars
% origin = "\cite{ref}"
%-----------------------------------------------------
\begin{Exercise}[
name={},
title={}, 
difficulty=0,
origin={\cite{JH}}]
    Show that if \( \vec{x}\dotprod\vec{y}=0 \) for every \( \vec{y} \)
    then \( \vec{x}=\zero \).
\end{Exercise}

\begin{Answer}
      We will prove this demonstrating that the contrapositive
      statement holds:~if \( \vec{x}\neq\zero \) then there
      is a \( \vec{y} \) with \( \vec{x}\dotprod\vec{y}\neq 0 \).

      Assume that \( \vec{x}\in\Re^n \).
      If \( \vec{x}\neq\zero \) then it has a nonzero component, say the
      \( i \)-th one \( x_i \).
      But the vector \( \vec{y}\in\Re^n \) that is all zeroes except for
      a one in component~$i$ gives
      \( \vec{x}\dotprod\vec{y}=x_i \).  
      (A slicker proof just considers $\vec{x}\dotprod\vec{x}$.)
\end{Answer}
