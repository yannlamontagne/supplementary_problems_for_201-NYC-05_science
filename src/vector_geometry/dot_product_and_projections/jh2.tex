%-----------------------------------------------------
% index key words
%-----------------------------------------------------
\index{orthogonal vector}

%-----------------------------------------------------
% name, leave blank
% title, if the exercise has a name i.e. Hilbert's matrix
% difficulty = n, where n is the number of stars
% origin = "\cite{ref}"
%-----------------------------------------------------
\begin{Exercise}[
name={},
title={}, 
difficulty=0,
origin={\cite{JH}}]
Is any vector orthogonal to itself?
\end{Exercise}

\begin{Answer}
Clearly \( u_1u_1+\cdots+u_nu_n \) is zero if and only if
      each \( u_i \) is zero.
      So only \( \zero\in\Re^n \) is orthogonal to itself.  
\end{Answer}
