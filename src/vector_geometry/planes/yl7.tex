%-----------------------------------------------------
% index key words
%-----------------------------------------------------
\index{plane}
\index{intersection, intersetion of two planes, intersection!planes}

%-----------------------------------------------------
% name, leave blank
% title, if the exercise has a name i.e. Hilbert's matrix
% difficulty = n, where n is the number of stars
% origin = "\cite{ref}"
%-----------------------------------------------------
\begin{Exercise}[
name={},
title={}, 
difficulty=0,
origin={\cite{YL}}]
Given two planes:
\[
\begin{array}{lllrrrrrrrrrrr}
\mathcal{P}_1 & : & \; &x&-&2y&+&z & = &1\\
\mathcal{P}_2 & : & \; &-4x&+&8y&-&4z & =&-4
\end{array}
\]
\Question Give a geometrical argument to explain why the intersection of the two planes is a plane.
\Question Find the solution set of the associated homogeneous linear system.
\Question Show that the solution set of the associated homegeneous linear system is orthogonal to the rows of the coefficient matrix of the system.
\Question Give a geometrical interpretation to part c).
\end{Exercise}

\begin{Answer}
\Question Both equations are multiples of each other, hence they have the same points in common.  Those points lie on both planes.
\Question $\rowvec{x,&y,&z}=s\rowvec{2,&1,&0)}+t\rowvec{-1,&0,&1}$ where $s,t\in\Re$
\Question $s\rowvec{2,&1,&0}+t\rowvec{-1,&0,&1}$ is othogonal to both $\rowvec{1,&-2,& 1}$ and $\rowvec{-4,&8,&-4}$.
\Question The rows of the coefficient matrix are the normals of the planes, hence the vectors that lie on the plane are orthogonal to the normals.
\end{Answer}
