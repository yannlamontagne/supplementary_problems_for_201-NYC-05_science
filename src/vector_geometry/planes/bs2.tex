%-----------------------------------------------------
% index key words
%-----------------------------------------------------
\index{plane}
\index{line}
\index{nearest point:to a point in a plane}

%-----------------------------------------------------
% name, leave blank
% title, if the exercise has a name i.e. Hilbert's matrix
% difficulty = n, where n is the number of stars
% origin = "\cite{ref}"
%-----------------------------------------------------
\begin{Exercise}[
name={},
title={}, 
difficulty=0,
origin={\cite{BS}}]
Given the lines $\mathcal{L}_1:\; (x, y, z) = (1, 3, 0) + t(4, 3, 1)$, $\mathcal{L}_2:\; (x, y, z) = (1, 2, 3) + t(8, 6, 2)$, and the plance $\mathcal{P}:\; 2x-y+3z=15$ and the point $A(1, 0, 7)$.
\Question Show that the lines $\mathcal{L}_1$ and $\mathcal{L}_2$ lie in the same plane and find the general equation of this plane.
\Question Find the point $B$ on the plane $\mathcal{P}$ which is closest to the point $A$.

\end{Exercise}
\begin{Answer}
\Question Since the two lines are parallel they lie on the same plane. $5x-6y-2x-13=0$.
\Question $B(-\frac{1}{7}, \frac{4}{7}, \frac{37}{7})$
\end{Answer}
